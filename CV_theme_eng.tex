\documentclass[10pt, letterpaper]{article}

% Packages:
\usepackage[
    ignoreheadfoot, % set margins without considering header and footer
    top=2 cm, % seperation between body and page edge from the top
    bottom=2 cm, % seperation between body and page edge from the bottom
    left=2 cm, % seperation between body and page edge from the left
    right=2 cm, % seperation between body and page edge from the right
    footskip=1.0 cm, % seperation between body and footer
    % showframe % for debugging 
]{geometry} % for adjusting page geometry
\usepackage[explicit]{titlesec} % for customizing section titles
\usepackage{tabularx} % for making tables with fixed width columns
\usepackage{array} % tabularx requires this
\usepackage[dvipsnames]{xcolor} % for coloring text

\definecolor{nameTagColor}{RGB}{100, 100, 100} 
\definecolor{primaryColor}{RGB}{155, 155, 155} % define primary color
\definecolor{headingOrange}{RGB}{244, 121, 32} % 

\usepackage{enumitem} % for customizing lists
\usepackage{fontawesome5} % for using icons
\usepackage{amsmath} % for math
\usepackage[
    pdftitle={DONGWOOK KIM's CV - \today},
    pdfauthor={DONGWOOK KIM},
    pdfcreator={LaTeX with RenderCV},
    colorlinks=true,
    urlcolor=nameTagColor
]{hyperref} % for links, metadata and bookmarks
\usepackage[pscoord]{eso-pic} % for floating text on the page
\usepackage{calc} % for calculating lengths
\usepackage{bookmark} % for bookmarks
\usepackage{lastpage} % for getting the total number of pages
\usepackage{changepage} % for one column entries (adjustwidth environment)
\usepackage{paracol} % for two and three column entries
\usepackage{ifthen} % for conditional statements
\usepackage{needspace} % for avoiding page brake right after the section title
\usepackage{iftex} % check if engine is pdflatex, xetex or luatex
\usepackage{graphicx}
\usepackage{tikz}

% Ensure that generate pdf is machine readable/ATS parsable:
\ifPDFTeX
    \input{glyphtounicode}
    \pdfgentounicode=1
    \usepackage[T1]{fontenc}
    \usepackage[utf8]{inputenc}
    \usepackage{lmodern}
\fi

\usepackage[default, type1]{sourcesanspro} 

\newcommand{\circularimage}[2][]{%
    \begin{tikzpicture}
        \node at (0,0) {\includegraphics[width=2.5cm, angle=0]{#2}};
    \end{tikzpicture}
}

% Some settings:
\AtBeginEnvironment{adjustwidth}{\partopsep0pt} % remove space before adjustwidth environment
\pagestyle{empty} % no header or footer
\setcounter{secnumdepth}{0} % no section numbering
\setlength{\parindent}{0pt} % no indentation
\setlength{\topskip}{0pt} % no top skip
\setlength{\columnsep}{0.15cm} % set column seperation
\makeatletter
\let\ps@customFooterStyle\ps@plain % Copy the plain style to customFooterStyle
\patchcmd{\ps@customFooterStyle}{\thepage}{
    \color{gray}\textit{\small DONGWOOK KIM - Page \thepage{} of \pageref*{LastPage}}
}{}{} % replace number by desired string
\makeatother
\pagestyle{customFooterStyle}

\titleformat{\section}{
    % avoid page braking right after the section title
    \needspace{4\baselineskip}
    % make the font size of the section title large and color it with the primary color
    \Large\color{primaryColor}
}{
}{
}{
    % print bold title, give 0.15 cm space and draw a line of 0.8 pt thickness
    % from the end of the title to the end of the body
    \textbf{#1}\hspace{0.15cm}\titlerule[0.8pt]\hspace{-0.1cm}
}[] % section title formatting

\titlespacing{\section}{
    % left space:
    -1pt
}{
    % top space:
    0.3 cm
}{
    % bottom space:
    0.2 cm
} % section title spacing

% \renewcommand\labelitemi{$\vcenter{\hbox{\small$\bullet$}}$} % custom bullet points
\newenvironment{highlights}{
    \begin{itemize}[
        topsep=0.10 cm,
        parsep=0.10 cm,
        partopsep=0pt,
        itemsep=0pt,
        leftmargin=0.4 cm + 10pt
    ]
}{
    \end{itemize}
} % new environment for highlights

\newenvironment{highlightsforbulletentries}{
    \begin{itemize}[
        topsep=0.10 cm,
        parsep=0.10 cm,
        partopsep=0pt,
        itemsep=0pt,
        leftmargin=10pt
    ]
}{
    \end{itemize}
} % new environment for highlights for bullet entries


\newenvironment{onecolentry}{
    \begin{adjustwidth}{
        0.2 cm + 0.00001 cm
    }{
        0.2 cm + 0.00001 cm
    }
}{
    \end{adjustwidth}
} % new environment for one column entries

\newenvironment{twocolentry}[2][]{
    \onecolentry
    \def\secondColumn{#2}
    \setcolumnwidth{\fill, 3.5 cm}
    \begin{paracol}{2}
}{
    \switchcolumn \raggedleft \secondColumn
    \end{paracol}
    \endonecolentry
} % new environment for two column entries

\newenvironment{twocolentry_project}[2][]{
    \onecolentry
    \def\secondColumn{#2}
    \setcolumnwidth{\fill, 2.0 cm}
    \begin{paracol}{2}
}{
    \switchcolumn \raggedleft \secondColumn
    \end{paracol}
    \endonecolentry
} % new environment for two column entries



\newenvironment{threecolentry}[3][]{
    \onecolentry
    \def\thirdColumn{#3}
    \setcolumnwidth{1 cm, \fill, 4.5 cm}
    \begin{paracol}{3}
    {\raggedright #2} \switchcolumn
}{
    \switchcolumn \raggedleft \thirdColumn
    \end{paracol}
    \endonecolentry
} % new environment for three column entries

\newenvironment{header}{
    \setlength{\topsep}{0pt}\par\kern\topsep\centering\color{nameTagColor}\linespread{1.5}
}{
    \par\kern\topsep
} % new environment for the header

\newcommand{\placelastupdatedtext}{% \placetextbox{<horizontal pos>}{<vertical pos>}{<stuff>}
  \AddToShipoutPictureFG*{% Add <stuff> to current page foreground
    \put(
        \LenToUnit{\paperwidth-2 cm-0.2 cm+0.05cm},
        \LenToUnit{\paperheight-1.0 cm}
    ){\vtop{{\null}\makebox[0pt][c]{
        \small\color{gray}\textit{Last updated in September 2024}\hspace{\widthof{Last updated in September 2024}}
    }}}%
  }%
}%

% save the original href command in a new command:
\let\hrefWithoutArrow\href

% new command for external links:
\renewcommand{\href}[2]{\hrefWithoutArrow{#1}{\ifthenelse{\equal{#2}{}}{ }{#2 }\raisebox{.15ex}{\footnotesize \faExternalLink*}}}


\begin{document}
    % --- 새로운 헤더 시작 (minipage 사용) ---
    \hspace{0.6cm} 
    \begin{minipage}[c]{3.5cm} % 왼쪽: 사진을 담을 상자, [c]는 세로 중앙 정렬 옵션
        \centering % 이미지 중앙 정렬
        \circularimage{profile_dwkim.png} % "profile.jpg"를 실제 파일명으로 변경
    \end{minipage}
    \hspace{0.5cm} % 사진과 글자 사이의 간격
    \begin{minipage}[c]{\linewidth-4cm} % 오른쪽: 글자를 담을 상자, 전체 폭에서 사진 상자 폭만큼 뺌
        \raggedright % 텍스트 왼쪽 정렬
        {\fontsize{30pt}{30pt}\color{nameTagColor}\textbf{DONGWOOK KIM}}
        
        \vspace{0.4cm}
        
        % 연락처 정보 (tabular 환경으로 줄 맞춤)
        \color{nameTagColor}
        \renewcommand{\arraystretch}{1.5}
        \begin{tabular}{@{} l @{}} % 왼쪽 정렬 테이블
            \faMapMarker* \hspace{0.2cm} Seoul, S.Korea \hspace{0.6cm} 
            \hrefWithoutArrow{mailto:donguk071@unist.ac.kr}{\faEnvelope[regular] \hspace{0.2cm} donguk071@unist.ac.kr} \hspace{0.6cm}
            \hrefWithoutArrow{tel:+82 10-5481-3292}{\faPhone* \hspace{0.2cm} +82 10-5481-3292} \hspace{0.6cm} \\
            \hrefWithoutArrow{https://donguk071.github.io/}{\faLink \hspace{0.2cm} Tech blog} \hspace{0.6cm} 
            \hrefWithoutArrow{https://github.com/donguk071}{\faGithub \hspace{0.2cm} donguk071} \hspace{0.6cm}
            \hrefWithoutArrow{https://www.linkedin.com/in/dongwook-kim-aab4142b8/}{\faLinkedinIn \hspace{0.2cm} DONGWOOK KIM}
        \end{tabular}
    \end{minipage}
    % --- 새로운 헤더 끝 ---

    

    \section{Interest}

    I am conducting research in 3D vision and AI agents, specializing in dataset distillation to enhance model training efficiency and scene reconstruction techniques such as Gaussian Splatting.
    Recently, I have been working on lightweight multimodal large language models (MLLMs) that incorporate point cloud understanding, as well as developing intelligent agent systems that integrate visual information with language-based reasoning.
    \vspace{0.2cm}


    \section{Education}
        \begin{twocolentry}{
            \small\textit{Ulsan, S.Korea}
            
            \vspace{0.2cm}
            
            \textit{Jan 2024 – Current}
        }
            % 기관명
            \textbf{\color{headingOrange}UNIST.}
            
            \vspace{0.2cm}
            
            % 학위 및 전공
            \textbf{Graduate School of Artificial Intelligence, AI core}
            \begin{highlights}
                \item Total GPA 4.2/4.3
                \item \hrefWithoutArrow{https://vip.unist.ac.kr/}{Visual Information Processing Lab \faLink}(Prof.Jae-Young Sim)
            \end{highlights}
        \end{twocolentry}

        \begin{twocolentry}{
            \small\textit{SanDiego, CA, US}
            
            \vspace{0.2cm}
            
            \textit{Jul 2022 – Aug 2022}
        }
            % 기관명
            \textbf{\color{headingOrange}Qualcomm Institute, UC San Diego.}
            
            \vspace{0.2cm}
            
            % 학위 및 전공
            \textbf{Qualcomm Institute AI Development Projects}
            \begin{highlights}
                \item International Researcher
            \end{highlights}
        \end{twocolentry}
        

        \begin{twocolentry}{
            \small\textit{Seoul, S.Korea}
            
            \vspace{0.2cm}
            
            \textit{Jan 2018 – 2024}
        }
            % 기관명
            \textbf{\color{headingOrange}Kwangwoon Univ.}
            
            \vspace{0.2cm}
            
            % 학위 및 전공
            \textbf{B.S in Information Convergence, Major in Data Science}
            \begin{highlights}
                \item Total GPA 3.98/4.5, Major GPA 4.16/4.5 (Credits taken: 116/133)
                \item \hrefWithoutArrow{https://korfriend.github.io/}{Deep Imaging and Graphics Lab \faLink}(Prof.Dongjoon Kim)
            \end{highlights}
        \end{twocolentry}
        
        \vspace{0.2 cm}


    
    \section{Publications}
    \subsection{International}


        \begin{twocolentry}{
            May 2026
        }
            \textbf{\color{headingOrange}[Under Review] ICLR 2026}

            \textbf{Parameterization-Based Dataset Distillation of 3D Point Clouds through Learnable Shape Morphing}
            (1st author) 

            \mbox{\textit{\underline{Dong-Wook Kim}, Jae-Young Yim  and Jae-Young Sim}}
            
            \vspace{0.10 cm}
            
        \end{twocolentry}
    
        \begin{twocolentry}{
            Nov 2025
        }
            \textbf{\color{headingOrange}NeurIPS 2025}

            \textbf{Dataset Distillation of 3D Point Clouds via Distribution Matching}
            (1st author, \hrefWithoutArrow{https://github.com/donguk071/SADM}{\faLink link}) 

            \mbox{\textit{Jae-Young Yim, \underline{Dong-Wook Kim} and Jae-Young Sim}}
            \vspace{0.10 cm}
            
        \end{twocolentry}


        \begin{twocolentry}{
            Sep 2025
        }
            \textbf{\color{headingOrange}ICIP-W 2025}

            \textbf{Class-Aware Coreset Selection for 3D Point Clouds Classification}
            (2nd author) 

            \mbox{\textit{Jae-Young Yim, \underline{Dong-Wook Kim}  and Jae-Young Sim}}
            
            \vspace{0.10 cm}
            
        \end{twocolentry}

        \subsection{Domestic}

        \begin{twocolentry}{
            Jun 2024
        }
            \textbf{\color{headingOrange}IPIU 2024}

            \textbf{Controllable Classification via Negative-Context-Aware Learning}
            (2nd author, \hrefWithoutArrow{http://www.ipiu.or.kr/images/mtl01r-20-0043/sub/IPIU2025_Program_v3.pdf}{\faLink link}) 

            \mbox{\textit{Jae-Young Yim, \underline{Dong-Wook Kim} and Jae-Young Sim}}
            
        
            \vspace{0.10 cm}
            
        \end{twocolentry}


        \begin{twocolentry}{
            Jun 2024
        }
            \textbf{\color{headingOrange}IEIE 2024}

            \textbf{Enhancing Quality of Gaussian Splatting in Few Shot Condition by Multi-Scale Augmentatio}
            (1st author, \hrefWithoutArrow{https://www.dbpia.co.kr/pdf/pdfView.do?nodeId=NODE11890955&width=1732}{\faLink link}) 

            \mbox{\textit{\underline{Dong-Wook Kim}, Jae-Young Yim and Jae-Young Sim}}
            
        
            \vspace{0.10 cm}
            
        \end{twocolentry}


        \begin{twocolentry}{
            Feb 2023
        }
            \textbf{\color{headingOrange}HCI Academy of Korea 2023}

            \textbf{Synthesized training data for a ship 3D surround view learning model based on user evaluations}
            (1st author, \hrefWithoutArrow{https://www.dbpia.co.kr/pdf/pdfView.do?nodeId=NODE11229776&width=1732}{\faLink link}) 
 
            \mbox{\textit{\underline{Dongwook Kim}, Jonghun Kim, Taemin Jeong and Dongjoon Kim}}
            
        
            \vspace{0.10 cm}
            
        \end{twocolentry}



        

            
    \section{Experience}

    \begin{twocolentry}{
        \small\textit{San Diego, CA, USA}
        
        \vspace{0.2cm}
        
        \textit{Jul. 2022 – Aug. 2022}
    }
        \textbf{\color{headingOrange}Qualcomm Institute, UC San Diego} \\
        \textbf{AI Development Project Intern}
        \begin{highlights}
            \item Developed a user classification model using the KNIME framework.
            \item Conducted research on preventing abusive behavior by analyzing and classifying user characteristics on Instagram.
        \end{highlights}
    \end{twocolentry}

    \begin{twocolentry}{
        \small\textit{Seoul, South Korea} 

        \textit{Jul. 2021 – Feb. 2024}
    }
        \textbf{\color{headingOrange}Kwangwoon University, Visual Informatics Lab} \\
        \textbf{Undergraduate Research Assistant}
        \begin{highlights}
            \item Conducted research on implicit modeling (NeRF, LIIF) and GAN-based 3D graphics.
            \item Led multiple projects involving LiDAR super-resolution, depth estimation, and 3D scene reconstruction.
        \end{highlights}
    \end{twocolentry}


    \vspace{0.2 cm}

    \section{Projects}


    \begin{twocolentry_project}{
        Nov. 2025
    }
        \textbf{Generative AI for Multimodal Transportation Analytics (with MSIT)}
        \begin{highlights}
            \item Developing an explainable multimodal chatbot for traffic analysis integrating CCTV video, sensor data, and LLM-RAG pipeline
            \item Building robust vision models for adverse-weather traffic scenes and constructing synthetic datasets via Unreal Engine simulation
            \item Designing a multimodal traffic knowledge graph and spatio-temporal GNNs for traffic prediction and anomaly detection
        \end{highlights}
    \end{twocolentry_project}

    \vspace{0.2 cm}


    \begin{twocolentry_project}{
        Jun. 2023
    }
        \textbf{Distortion-Free SVM Generation Project(with Avicus)}
        \begin{highlights}
            \item Built a real-time digital twin of a marine environment using Unreal Engine 5, generating synthetic data from virtual LiDAR and RGB sensors.
            \item Developed deep learning models for sensor data enhancement, including LiDAR super-resolution and image segmentation.
            \item Integrated a 3D geometry reconstruction pipeline using calibrated sensor data to achieve a distortion-free Surround View Monitoring (SVM) system.
        \end{highlights}
    \end{twocolentry_project}

    \vspace{0.2 cm}
    \begin{twocolentry_project}{Dec. 2022}
        \textbf{VR : Anchorage Simulation with Unreal Engine 5}
        \begin{highlights}
            \item Built boat navigation/docking simulator using Unreal Engine 5 (C++/Blueprints), Oculus SDK/OpenXR.
            \item Implemented hand–gesture steering via UE Motion Controller.
        \end{highlights}
    \end{twocolentry_project}
    
    \vspace{0.2 cm}
    \begin{twocolentry_project}{Jun. 2022}
        \textbf{AR : Distortion-Free SVM Generation Project (with Avicus)}
        \begin{highlights}
            \item Real-time face capture \& avatar rigging with MediaPipe Face Mesh, Three.js/WebGL, blendshape mapping(landmark)
            \item Optimized animation with inverse kinematics (IK) and temporal smoothing for stable expression control
        \end{highlights}
    \end{twocolentry_project}
        
    \vspace{0.2 cm}
    \begin{twocolentry_project}{Jul. 2022}
        \textbf{ML/DL : Dementia Prediction Project}
        \begin{highlights}
            \item Predicted dementia risk using tabular data and wearable signal features via ensemble ML models
            \item Applied DL-based signal classification and feature engineering for multimodal health data
        \end{highlights}
    \end{twocolentry_project}

    \vspace{0.2 cm}
    \begin{twocolentry_project}{Current}
        \textbf{Computer Vision : OpenCV Camera Calibration \& Panda3D XR Project}
        \begin{highlights}
            \item Implementing camera calibration, pose estimation, and XR visualization using OpenCV \& Panda3D
            \item Developing real-time calibration pipeline for stereo vision XR systems
        \end{highlights}
    \end{twocolentry_project}

    \vspace{0.2 cm}
    \section{Competition \& Awards}

    \begin{twocolentry_project}{
        Oct. 2025
    }
        \textbf{2nd Prize, 2025 Samsung AI Challenge}
        \begin{highlights}
            \item Developed an AI co-researcher agent integrating MCP and RAG architectures.
            \item Implemented context management using ChromaDB for efficient information retrieval and dialogue grounding.
        \end{highlights}
    \end{twocolentry_project}

    \vspace{0.2 cm}

    \begin{twocolentry_project}{
        Feb. 2024
    }
        \textbf{1st Prize, SKT FLY AI Competition}
        \begin{highlights}
            \item Developed a motion-synchronized meta character that reacts dynamically to conversation using rigging, retargeting, and TTS.
            \item Designed and trained an emotion classification network through conversational text data.
        \end{highlights}
    \end{twocolentry_project}

    \vspace{0.2 cm}

    \begin{twocolentry_project}{
        Oct. 2022
    }
        \textbf{1st Prize (Minister’s Award), AI Contest for Software-Centered Universities}
        \begin{highlights}
            \item Competed in an OCR task for signage image recognition.
            \item Enhanced model performance through advanced data augmentation and ensemble learning strategies.
        \end{highlights}
    \end{twocolentry_project}

    \vspace{0.2 cm}

    \begin{twocolentry_project}{
        Jun. 2022
    }
        \textbf{3rd Prize, Student Creative Design Competition}
        \begin{highlights}
            \item Developed \textit{Coverist}, an AI-based book cover generation service.
            \item Led the end-to-end development process, including AI modeling, web backend, and mobile deployment.
        \end{highlights}
    \end{twocolentry_project}

    \vspace{0.2 cm}




    % \section{Patent}    
    
    % \section{Extra Activity}    


    \section{Technologies}

        \begin{onecolentry}
            \textbf{Programming Languages:} Python, C++, JavaScript, Unreal Blueprints
        \end{onecolentry}
    
        \vspace{0.2 cm}
    
        \begin{onecolentry}
            \textbf{Deep Learning \& Vision:} TensorFlow, PyTorch, OpenCV
        \end{onecolentry}
        
        \vspace{0.2 cm}
    
        \begin{onecolentry}
            \textbf{Simulation \& Graphics:} Unreal Engine, Three.js, Panda3D
        \end{onecolentry}


        
\end{document}